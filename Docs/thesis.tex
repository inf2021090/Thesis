\documentclass[a4paper,11pt,oneside,openany]{ioniothesis}

\usepackage{biblatex} %Imports biblatex package
\addbibresource{references.bib} %Import the bibliography file


\usepackage[dvips]{graphicx}
\usepackage{makeidx}
\usepackage{booktabs,calc,multirow}
\usepackage{alphabeta}
%\usepackage[greek,english]{babel}
\usepackage[utf8]{inputenc}
\usepackage{amssymb}
\usepackage{amsfonts}
\usepackage{color}
\usepackage{ioniostyle}
\usepackage{float}
\usepackage{listings}

\lstset{language=python,
	frame=tb,
	keywordstyle=\bfseries\ttfamily\color[rgb]{0,0,1},
	identifierstyle=\ttfamily,
	commentstyle=\color[rgb]{0.133,0.545,0.133},
	stringstyle=\color[rgb]{0.8,0,0},
	showstringspaces=false,
	basicstyle=\small,
	numberstyle=\footnotesize,
	numbers=left,
	stepnumber=1,
	numbersep=10pt,
	tabsize=4,
	breaklines=true,
	prebreak = \raisebox{0ex}[0ex][0ex]{\ensuremath{\hookleftarrow}},
	breakatwhitespace=false,
	aboveskip={0.5\baselineskip},
  columns=fixed,
  upquote=true,
  extendedchars=true,
}

\def\tl{\textlatin}
\def\tg{\textgreek}

\parindent=0pt

\makeindex
\evensidemargin=0 cm
\oddsidemargin=1 cm
\textwidth=15 cm


%The following line is for one and half spacing
%\linespread{1.3}
%The following line is for double spacing
%\linespread{1.6}
\linespread{1.3}

\usepackage{drop}



\begin{document}




\author{\textbf{Όνοματεπώνυμο Φοιτητή/τριας} \\ \textbf{Επιβλέπων:}  -- Όνομα Επιβλέποντος --}
\title{
\textbf{\LARGE{\textsc{ΙΟΝΙΟ ΠΑΝΕΠΙΣΤΗΜΙΟ}} 
\bigskip \\
\large{\textsc{Τμήμα Πληροφορικής}}
\bigskip \\ \bigskip \bigskip \bigskip
\bigskip \bigskip 
\includegraphics[width=0.7\textwidth]{./pdffigs/ionian-university.jpg}
\bigskip \\ \bigskip 
\texttt{\large{-- Πτυχιακή Εργασία --}}
\bigskip \\ \bigskip
\textbf{\Large{\texttt{
Τίτλος Πτυχιακής
}}}
\bigskip \\ \bigskip}
}
\maketitle



\pagestyle{empty}
\cleardoublepage



\chapter*{}
\thispagestyle{empty}

\begin{center}
\Large{\textbf{Επιβλέπων(ουσα)}}
\end{center}
\prof{Ονοματεπώνυμο}{Βαθμίδα}{Ίδρυμα}

\begin{center}
\Large{\textbf{Τριμελής Επιτροπή}}
\end{center}
\prof{Ονοματεπώνυμο}{Βαθμίδα}{Ίδρυμα}
\prof{Ονοματεπώνυμο}{Βαθμίδα}{Ίδρυμα}
\prof{Ονοματεπώνυμο}{Βαθμίδα}{Ίδρυμα}





\pagestyle{empty}
\cleardoublepage


\cleardoublepage

  
\pagenumbering{roman}


%Ðåñßëçøç
\chapter*{Περίληψη} \pagestyle{headings}
\input{perilipsi}
\cleardoublepage

%Ðñüëïãïò êáé Åõ÷áñéóôßåò
\chapter*{Πρόλογος και Ευχαριστίες} \pagestyle{headings}
\input{prologos}
\cleardoublepage

\renewcommand*\contentsname{Περιεχόμενα}
\renewcommand*\listfigurename{Κατάλογος Σχημάτων}
\renewcommand*\listtablename{Κατάλογος Πινάκων}
\renewcommand\bibname{Βιβλιογραφία}
\renewcommand\chaptername{Κεφάλαιο}
\renewcommand\tablename{Πίνακας}
\renewcommand\figurename{Σχήμα}

\tableofcontents

\cleardoublepage

\listoffigures
\cleardoublepage
\listoftables

\setlength{\parskip}{5pt}



\pagestyle{headings}
\cleardoublepage


\newpage
\pagenumbering{arabic}


\cleardoublepage


\chapter{Εισαγωγή} \label{chapter:intro}

\input{01 - intro}


\chapter{Θεμελιώδεις Έννοιες και Αρχές Μοντελοποίησης Βιολογικών Δικτύων} \label{chapter:kefalaio-b}

\input{02- kefalaio-b}

\chapter{Μοντελοποίηση Βιολογικών Δικτύων με Μεθόδους Μηχανικής Μάθησης} \label{chapter:kefalaio-c}

\input{03- kefalaio-c}

%Bibliography


\printbibliography %Prints bibliography


%CHAPTER
\chapter*{Συντμήσεις} \pagestyle{empty}
\addcontentsline{toc}{chapter}{Συντμήσεις}
\input{abbrev}


%CHAPTER
\chapter*{Γλωσσάρι Ξενικών Όρων} \pagestyle{empty}
\addcontentsline{toc}{chapter}{Γλωσσάρι Ξενικών Όρων}
\input{glossari}





\newpage
\addcontentsline{toc}{chapter}{Ευρετήριο}
\printindex



\end{document}
